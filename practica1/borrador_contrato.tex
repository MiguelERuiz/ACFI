\documentclass[a4paper,11pt]{report}

%%%%%%%%%%%%%%%%%%%%%%%%%%%%%%%%%%%%%%%%%%%%%%%%%%%%%%%%%%%%%%%%%%%%%%%
% Definicion de paquetes
\usepackage[utf8]{inputenc}

%%%%%%%%%%%%%%%%%%%%%%%%%%%%%%%%%%%%%%%%%%%%%%%%%%%%%%%%%%%%%%%%%%%%%%%
%% Empieza el documento
\begin{document}
\title{BORRADOR DE CONTRATO DE DESARROLLO SOFTWARE}
\author{
  Sergio García Sánchez
  \and
  Sara Juberías Campos
  \and
  José Luis Mela Navarro
  \and
  Amalia Regueira Fernández
  \and
  Miguel Emilio Ruiz Nieto
}

\maketitle

REUNIDOS

DE UNA PARTE,

\_\_\_\_\_\_\_\_, mayor de edad, con domicilio en \_\_\_\_\_\_\_\_, DNI/NIF núm.
\_\_\_\_\_\_\_\_, y en su propio nombre y representación.

En adelante, el "DESARROLLADOR".

DE OTRA PARTE,

\_\_\_\_\_\_\_\_, mayor de edad, con domicilio en \_\_\_\_\_\_\_\_, DNI/NIF núm.
\_\_\_\_\_\_\_\_, y en su propio nombre y representación.

En adelante, el "CLIENTE".

El Desarrollador y el Cliente que, en adelante, podrán ser denominados,
individualmente, la "PARTE" y conjuntamente, las "PARTES",
reconociéndose mutuamente la capacidad jurídica necesaria para contratar
y obligarse, y en especial, para el otorgamiento del presente CONTRATO
DE DESARROLLO DE SOFTWARE. En adelante, el "CONTRATO".

EXPONEN

I. Que el Desarrollador se dedica a la siguiente actividad:

\_\_\_\_\_\_\_\_

II. Que en virtud de lo anterior, el Desarrollador dispone de los
conocimientos y medios necesarios para el diseño y codificación de
programas informáticos o aplicaciones;

III. Que el Cliente desea el desarrollo de un programa informático (en
adelante, el "SOFTWARE"), con una serie de características y
funcionalidades determinadas que han sido comunicadas al Desarrollador;

IV. Que, con el fin de desarrollar el Software de acuerdo con las
especificidades y funcionalidades solicitadas por el Cliente, las Partes
han negociado y aceptado un presupuesto y un plan de trabajo para el
desarrollo del Software;

V. Que, en virtud de lo anterior, el Desarrollador desea, libre y
espontáneamente, comprometerse a diseñar, estructurar y codificar el
Software atendiendo a las instrucciones del Cliente, operación que las
Partes desean formalizar a través del presente Contrato, que se regirá
por las siguientes,

ESTIPULACIONES

PRIMERA. OBJETO DEL CONTRATO

Mediante el presente Contrato, el Desarrollador se compromete a diseñar,
estructurar y codificar el Software en favor del Cliente dentro del
plazo fijado en el presente Contrato.

De esta forma, el Desarrollador creará un Software personalizado según
el presupuesto o los documentos técnicos negociados entre las Partes de
forma previa a la firma del presente Contrato, cuya titularidad
corresponderá en su integridad al Cliente.

Por último, el diseño, la estructuración y codificación del Software se
ceñirá a lo establecido en las estipulaciones de este Contrato y a lo
dispuesto en el Real Decreto Legislativo 1/1996, de 12 de abril, por el
que se aprueba el texto refundido de la Ley de Propiedad Intelectual,
así como a la restante legislación aplicable.

SEGUNDA. FUNCIONALIDADES DEL SOFTWARE

Siguiendo las peticiones del Cliente, y de acuerdo con la negociación
previa llevada a cabo entre las Partes, el Desarrollador queda obligado
a diseñar y codificar el Software cumpliendo con las siguientes
funcionalidades o características:

\_\_\_\_\_\_\_\_

El Desarrollador diseñará y codificará el Software siguiendo lo
establecido en esta Estipulación, así como lo recogido en el propio
Contrato y en todos sus Anexos.

TERCERA. PRECIO Y FORMA DE PAGO

Las Partes acuerdan el pago de una cantidad ascendiente a \_\_\_\_\_\_\_\_
(\_\_\_\_\_\_\_\_€) como remuneración del Software, sin incluir los impuestos
que se pudieran derivar de esta operación.

El Cliente satisfará el pago del precio fijo el \_\_\_\_\_\_\_\_ mediante pago
en efectivo en favor del Desarrollador.

Por último, el Desarrollador emitirá una factura al Cliente cumpliendo
con los requisitos legales necesarios y dentro de los plazos previstos
en la legislación actual.

CUARTA. INTERESES DE DEMORA

Siguiendo lo recogido en el artículo 1.101 del Código Civil, cualquier
retraso en el pago de la remuneración establecida en el presente
Contrato dará lugar a un incremento del precio equivalente a los
intereses de demora generados por el retraso en el pago.

El tipo de interés de demora será igual al tipo de interés de referencia
o de refinanciación semestral del Banco Central Europeo en vigor a 1 de
enero para el primer semestre del año correspondiente, y a 1 de julio
para el segundo semestre del año correspondiente.

Los intereses de demora serán exigibles automáticamente a partir de la
fecha de pago fijada en la Estipulación anterior, sin necesidad alguna
de aviso del vencimiento ni intimación alguna por parte del
Desarrollador. El devengo de dichos intereses no afectará al ejercicio
de cualquier acción que pueda corresponderle al Desarrollador derivada
del incumplimiento del pago.

QUINTA. ORIGINALIDAD DEL SOFTWARE

El Desarrollador manifiesta de forma expresa que el Software será
íntegramente diseñado y codificado por el Desarrollador, por su personal
cualificado, o por colaboradores en línea con lo dispuesto en este
Contrato, respetando, en su integridad, la legislación sobre la
propiedad intelectual aplicable.

En virtud de lo anterior, el Desarrollador manifiesta que no utilizará
ningún código u cualquier otro tipo de propiedad intelectual de terceros
distintos a los citados en el párrafo anterior, ya sea un código fuente
o un código de carácter complementario, para el desarrollo del Software.
El Software será totalmente codificado de forma original por el
Desarrollador y sus colaboradores, en su caso.

Siguiendo estas manifestaciones, el Desarrollador se compromete a:

A. Exonerar al Cliente de toda responsabilidad frente a terceros que
aleguen una violación de sus derechos de propiedad intelectual que
puedan tener sobre el Software.

B. Garantizar el uso del Software, debiendo indemnizar al Cliente en el
caso de que por resolución judicial se impida su uso por causas o
motivos generados con anterioridad a la entrega del Software en favor
del Cliente.

Además, las Partes acuerdan que por razones de equidad, y en la medida
que lo justifiquen las circunstancias, el Cliente podrá reclamar la
restitución de las sumas pagadas o gastos derivados del Contrato hasta
la fecha en la que tiene lugar el pronunciamiento judicial en favor de
un tercero o en la que se produce la cancelación o denegación de la
inscripción del Software en el registro de la propiedad intelectual
correspondiente.

C. Mantener informado al Cliente de todos los posibles usos fraudulentos
o violaciones del Software que hayan podido realizar terceros durante el
desarrollo del Software, comprometiéndose a adoptar todas las medidas
necesarias para garantizar su protección y permitir el correcto uso del
Software.

D. Garantizar que, en caso de ser necesario, el Cliente cuenta con el
consentimiento de todos desarrolladores, incluidos sus posibles
colaboradores, para la cesión de su posible propiedad sobre el Software
en favor del Cliente, recayendo bajo su responsabilidad los perjuicios
que se puedan derivar de la ausencia del mismo.

SEXTA. PLAZO DE EJECUCIÓN

El Contrato entrará en vigor en la fecha señalada en el encabezado del
presente Contrato. El Desarrollador deberá llevar a cabo el diseño y
codificación del Software siguiendo los plazos establecidos a
continuación:

- \_\_\_\_\_\_\_\_, se deberá completar lo siguiente:

\_\_\_\_\_\_\_\_

En todo caso, el Cliente se compromete a colaborar y a aportar toda la
información necesaria que le sea requerida por el Desarrollador para
poder codificar el Software de acuerdo a los plazos establecidos en esta
Estipulación. En el caso de que el Cliente no facilite esta información
o no preste su colaboración de forma adecuada, el Desarrollador podrá
comunicar por escrito al Cliente su imposibilidad de cumplir con los
plazos establecidos en esta Estipulación.

SÉPTIMA. MODIFICACIÓN DEL SOFTWARE

El Cliente podrá solicitar al Desarrollador la introducción de cambios,
modificaciones o mejoras en el Software durante su desarrollo y antes,
en todo caso, de su finalización.

Estos cambios se acordarán por escrito y serán adoptados por el
Desarrollador, debiéndose en todo caso negociar los nuevos plazos de
entrega y el precio según las modificaciones solicitadas.

OCTAVA. PROPIEDAD DEL SOFTWARE

El Desarrollador reconoce de forma expresa que el Cliente será el
titular de pleno dominio de la propiedad intelectual del Software. De
esta forma, el Desarrollador renuncia, expresa y totalmente, a cuantos
derechos de propiedad intelectual pudieran generarse como consecuencia
del desarrollo del Software.

En consecuencia, el Desarrollador se obliga a no hacer uso de estos
derechos para fines distintos a los del cumplimiento del Contrato. El
resultado de los trabajos realizados, en su totalidad o en cualquiera de
sus fases, será propiedad del Cliente y este, en consecuencia, podrá
solicitar en cualquier momento la entrega de los documentos o materiales
que la integren, con todos sus antecedentes, borradores, datos o
procedimientos.

No obstante, la cesión de los derechos de explotación de software o
programas informáticos preexistentes de propiedad del Desarrollador que
el Cliente estuviera interesado en incorporar como parte del futuro
Software serían negociados expresamente, y caso por caso, entre el
Desarrollador y el Cliente. El Desarrollador informaría previamente al
Cliente de forma clara y concisa sobre las condiciones de adquisición
y/o explotación, para que éste pueda decidir libremente sobre los
mismos.

El Cliente podrá libremente decidir llevar a cabo la inscripción del
Software o no en el Registro de la Propiedad Intelectual
correspondiente, así como realizar todas las renovaciones de dicha
inscripción que se considere necesarias.

La cesión de los derechos de propiedad intelectual en favor del Cliente
prevista en el presente Contrato no tiene ámbito geográfico determinado;
es decir, se extiende a todos los países del mundo sin limitación
geográfica de ninguna clase.

Por último, el Desarrollador no mantendrá derecho de propiedad alguno,
ni tendrá ningún derecho de compensación más allá de lo establecido en
este Contrato, sobre todas aquellas partes del Software o de sus módulos
desarrollados.

NOVENA. CESIÓN DEL CONTRATO. SUBCONTRATACIÓN

Las Partes no podrán ceder su posición en el presente Contrato, ni
tampoco los derechos u obligaciones que de este mismo emanasen a su
favor o a su cargo, sin el consentimiento previo, expreso y por escrito
de la otra Parte.

En particular esta Estipulación regirá de forma que el Desarrollador se
compromete a comunicar al Cliente por escrito, y de forma previa a la
celebración de un acuerdo de subcontratación, su intención de contratar
a una parte subcontratista o colaboradora, la identidad de la misma, el
tipo de servicios y trabajos a realizar por esta en referencia al
Software y las condiciones económicas y legales, de la relación de
subcontatación. Todo ello en orden de facilitar que el Cliente pueda
aprobar dicha subcontratación, sin que dicha autorización suponga la
asunción de responsabilidad alguna por parte del Cliente o la aprobación
del resultado de los servicios y trabajos que la parte subcontatista o
colaboradora provea.

Igualmente, será responsabilidad del Desarrollador comprobar que la
parte subcontratista está autorizada para la prestación de los servicios
o actividades objeto de subcontratación, así como regular por escrito la
relación contractual con la misma, incorporando o anexando los acuerdos
establecidos en el presente Contrato. Además, el Desarrollador se obliga
a entregar al Cliente una copia de dicho contrato en los 5 (cinco) días
siguientes a su firma.

La parte subcontratista actuará en todo momento bajo la dirección y
control del Desarrollador, quien se obliga y responsabiliza de hacer
cumplir todas las obligaciones asumidas en el presente Contrato.

El Desarrollador responderá solidariamente de las obligaciones que asuma
la parte subcontratista, incluso cuando el Cliente hubiera autorizado
dicha subcontratación, incluyendo los daños y perjuicios que pudiese
sufrir directa o indirectamente por la actuación de dicha parte
subcontratista. Del mismo modo, cualquier acto, error o negligencia en
el cumplimiento de las obligaciones laborales o de Seguridad Social de
la parte subcontratista, de sus representantes, o trabajadores, no
serán, en ningún caso, responsabilidad del Cliente.

El incumplimiento de esta Estipulación por el Desarrollador será motivo
suficiente para resolver el presente Contrato.

DÉCIMA. INEXISTENCIA DE RELACIÓN LABORAL

Las Partes declaran de forma expresa que, a todos los efectos legales,
el Desarrollador desempeña su actividad de forma totalmente
independiente, siguiendo su propia organización y con sus medios
personales y técnicos.

De esta forma, las Partes reconocen la inexistencia de relación laboral
alguna entre ellas según lo establecido en el artículo 1 y siguientes
del Real Decreto Legislativo 2/2015, de 23 de octubre, por el que se
aprueba el texto refundido de la Ley del Estatuto de los Trabajadores.

En virtud de lo anterior, no será de aplicación la normativa laboral ni
cabrá la posibilidad por ninguna de las Partes de establecer reclamación
alguna en esta materia.

DECIMOPRIMERA. CUMPLIMIENTO NORMATIVA APLICABLE

El Desarrollador se compromete a desarrollar el Software cumpliendo de
forma diligente con toda la normativa aplicable y, en particular, con
todas las obligaciones laborales, de la Seguridad Social, fiscales y de
protección de datos que le sean aplicables en relación con el desarrollo
del Software.

DECIMOSEGUNDA. ENTREGA DEL SOFTWARE

Una vez cumplido el plazo fijado en la Estipulación Sexta del presente
Contrato, se realizará la entrega del Software en favor del Cliente. La
entrega del Software se deberá llevar a cabo de la siguiente forma:

\_\_\_\_\_\_\_\_

Por otro lado, el Cliente se compromete a la recepción del Software,
pudiendo en todo caso expresar las reservas o incidencias que estime
oportunas sobre el mismo en el caso de que no satisfaga sus
expectativas.

Si las Partes así lo acuerdan, se podrá aprobar la entrega parcial antes
del vencimiento del plazo antes mencionado de partes o módulos del
Software que puedan funcionar de forma autónoma.

DECIMOTERCERA. PERÍODO DE PRUEBA

Tras la entrega del Software, las Partes acuerdan un período de prueba
del Software de una duración: \_\_\_\_\_\_\_\_. Este plazo comenzará a contar
desde el momento de entrega del Software en favor del Cliente según lo
establecido en la Estipulación anterior.

Durante este período de prueba, el Cliente llevará a cabo todos los
tests o pruebas que sean necesarios a fin de detectar todos los posibles
errores o fallos técnicos del código, así como de garantizar la correcta
adaptación del Software a los sistemas informáticos del Cliente. El
Cliente comunicará al Desarrollador todos los posibles fallos técnicos
detectados con el fin de permitir su reparación.

Una vez depurados y solventados todos estos errores o fallos técnicos, o
superado el plazo indicado sin que el Cliente manifieste la existencia
de fallo alguno, se entenderá por realizada la entrega del Software a
todos los efectos.

DECIMOCUARTA. GARANTÍA DEL SOFTWARE

Tras la realización de la entrega del Software, y superado el periodo de
prueba, serán de aplicación al mismo las garantías previstas en el
Código Civil, así como lo dispuesto en el texto refundido de la Ley
General para la Defensa de Consumidores y Usuarios y en el Código de
Comercio en su caso.

En concreto, las Partes acuerdan un periodo de garantía con una duración
de: \_\_\_\_\_\_\_\_ desde la entrega del Software.

Por último, la garantía legal no será de aplicación en el caso de que el
Cliente incumpla con su obligación de pago establecida en la
Estipulación tercera o decida terminar de forma anticipada este Contrato
por decisión voluntaria.

DECIMOQUINTA. GASTOS

El Desarrollador se hará cargo de todos los gastos derivados en relación
con el diseño y codificación del Software. Deberá hacerse cargo del pago
de los medios e instrumentos necesarios para poder ejecutar el Contrato
correctamente, así como de todos los impuestos o tasas que se devenguen
en relación con la creación del Software, quedando el Cliente
completamente indemne del pago de todos estos gastos.

DECIMOSEXTA. ELEVACIÓN A PÚBLICO DEL CONTRATO

Cualquiera de las Partes podrá solicitar, mediante requerimiento
fehaciente, la elevación a público del presente Contrato.

En ese caso, las Partes elegirán por mutuo acuerdo el Notario o Notaria
Público ante el cual se otorgará la escritura pública y la parte
solicitante se hará cargo de los correspondientes gastos notariales.

DECIMOSÉPTIMA. FUERZA MAYOR

El retraso en el cumplimiento de cualquier obligación de las Partes no
será considerado una omisión o un incumplimiento del Contrato en el caso
de que tenga su origen en causas imprevisibles o inevitables (en
adelante, "FUERZA MAYOR"), siempre que se haya informado debidamente a
la otra Parte sobre esta situación.

Se entenderá como Fuerza Mayor, entre otras: inundación, incendio,
explosión, avería en la planta de producción, cierre patronal, huelga,
disturbio civil, bloqueo, embargo, mandato, ley, orden, regulación,
ordenanza, demanda o petición del gobierno, o cualquier otra causa que
se encuentre fuera de control de la Parte involucrada, sin que pueda
entenderse que la falta de fondos constituye una causa de Fuerza Mayor.

La Parte afectada por la Fuerza Mayor hará todo lo posible por eliminar
su causa. La exigibilidad de la obligación cuyo cumplimiento se haya
visto afectado por la situación de Fuerza Mayor se suspenderá hasta diez
(10) días después de que la situación de Fuerza Mayor deje de impedir o
retrasar el cumplimiento. Si la causa de Fuerza Mayor no desaparece tras
treinta (30) días, o un plazo inferior que justifique la resolución por
la imposibilidad de ejecutar la obligación, las Partes podrán acordar la
modificación del Contrato o su resolución.

DECIMOCTAVA. OBLIGACIÓN DE NO COMPETENCIA

El Desarrollador se compromete a no tener o mantener, ni directa ni
indirectamente, intereses o a participar y desarrollar actividades,
proyectos o empresas que supongan una competencia directa o indirecta
con el uso, explotación o desarrollo del Software. Esta obligación de no
competencia se extenderá en el siguiente territorio o zona geográfica:
\_\_\_\_\_\_\_\_.

Así, durante un plazo de \_\_\_\_\_\_\_\_ desde la fecha recogida en el
encabezado de este Contrato, el Desarrollador no podrá, ya sea de forma
directa o indirecta, compartir ningún tipo de información a la que
pudiesen haber tenido acceso, ni ser propietarios, dirigir, controlar,
participar, como inversores, directivos, consultores o consejeros o de
cualquier otra manera, ser contratados, o contratar a empleados del
Cliente, o captar clientes de la misma, para o con beneficio a una
sociedad competidora en la comercialización o gestión del Software. Ya
sea en régimen de contrato laboral, incluso si se trata de una relación
laboral especial, o en régimen de arrendamiento de servicios o
asesoramiento interno, de modo directo o indirecto a través de personas
jurídicas interpuestas.

El Desarrollador reconoce que la prohibición de no competencia
post-contractual se encuentra debidamente compensada dentro del precio
del desarrollo del Software.

En todo caso, el Cliente podrá eximir de la obligación de no competencia
al Desarrollador cuando entienda que su actividad no supondría un
perjuicio.

No obstante, a la vista del daño que se causaría en el caso de
incumplimiento de la presente Estipulación, el Desarrollador deberá
abonar inmediatamente al Cliente una indemnización compensatoria
razonable y proporcional al daño que se cause.

Esta penalidad se considera justa por parte del Desarrollador dada la
relevancia para la actividad del Cliente que supone el respeto de lo
contenido en esta Estipulación.

DECIMONOVENA. OBLIGACIÓN DE SECRETO Y CONFIDENCIALIDAD

Las Partes reconocen que toda la información a la que se pueda tener
acceso en el marco del Contrato, ya sea relacionada con el Software
objeto de desarrollo o relacionada con la actividad u organización de
alguna de las Partes (en adelante, la "INFORMACIÓN"), tiene carácter
confidencial. De esta forma, las Partes acuerdan no divulgarla y
mantener la más estricta confidencialidad respecto de dicha Información,
advirtiendo, en su caso, de dicho deber de confidencialidad y secreto a
sus empleados, asociados y a cualquier persona que, por su cargo o
relación personal o sentimental deba o pueda tener acceso a la misma.

Ninguna de las Partes podrá reproducir, modificar, hacer pública o
divulgar a terceros la Información sin previa autorización escrita y
expresa de la otra Parte.

Las Partes se comprometen a poner los medios necesarios para que la
Información no sea divulgada ni cedida. Adoptarán las mismas medidas de
seguridad que adoptarían respecto a la información confidencial de su
propiedad, evitando su pérdida, robo o sustracción.

El receptor de la Información se compromete, en su caso, a advertir
sobre la existencia del deber de confidencialidad a sus empleados,
asociados, y a toda persona a la cual se le facilite la Información,
haciéndose responsable del uso indebido que estos puedan hacer de la
Información relacionada con el Contrato.

Asimismo, la Parte que recibe la Información se compromete a poner en
conocimiento de la otra Parte cualquier acción o incidente por parte de
terceros que pueda atentar contra la confidencialidad de la Información.

Ambas Partes se comprometen a que la utilización de la Información solo
estará dirigida a alcanzar los objetivos del Contrato y no otros, y que,
así, solo estará en conocimiento de aquellas personas estrictamente
necesarias para cumplir con aquellos.

Las disposiciones relativas a la confidencialidad previstas en este
Contrato se aplicarán durante la vigencia del mismo y, prevalecerán
durante el siguiente período: \_\_\_\_\_\_\_\_ tras su terminación. Este plazo
de tiempo es inmediato a la terminación del Contrato.

VIGÉSIMA. INEXISTENCIA DE RENUNCIA

La renuncia de una de las Partes a exigir el cumplimiento de alguna de
las obligaciones previstas en el Contrato, o a ejercer alguno de los
derechos o acciones que le asisten en virtud del mismo, (A) no liberará
a la otra Parte del cumplimiento íntegro de las restantes obligaciones
contenidas en el Contrato; y, (B) no se entenderá como una renuncia a
exigir en un futuro el cumplimiento de cualquier obligación o a ejercer
derechos o acciones previstos en el Contrato.

La dispensa, aplazamiento o renuncia de alguno de los derechos
contemplados en el Contrato, o a una parte de los mismos, será
únicamente vinculante si consta por escrito, pudiendo quedar sujeta a
las condiciones que el otorgante de dicha dispensa, aplazamiento o
renuncia considere oportuno, limitándose al caso concreto en el que se
produjo, y no restringirá, en ningún caso, su exigibilidad en otros
supuestos del derecho al que afecta.

VIGESIMOPRIMERA. TERMINACIÓN ANTICIPADA DEL CONTRATO

El presente Contrato podrá ser resuelto por el mutuo acuerdo de las
Partes, con los efectos que ellas determinen. En todo caso, la
terminación del Contrato se deberá formular por escrito.

Por otro lado, el Contrato finalizará en el transcurso del periodo de
duración inicial, o de cualquiera de sus prórrogas, siempre que
cualquiera de las Partes denuncie su prórroga de acuerdo a lo
establecido en la Estipulación de Duración del Contrato.

Igualmente, podrá ser resuelto en cualquier momento por cada una de las
Partes, a su elección, sin necesidad de intervención judicial, y sin
perjuicio de la responsabilidad en la que incurra la otra Parte por su
incumplimiento contractual, siempre que existan "causas justificadas",
tal y como se expone a continuación:

A. el incumplimiento total o parcial por la otra Parte de alguna de las
condiciones u obligaciones esenciales de este Contrato que no sea
corregido en el plazo de diez (10) días a partir de la notificación
escrita y fehaciente para que así lo haga; y,

B. las demás establecidas en el articulado del presente Contrato o las
que se recojan en la ley, y en concreto, en el Código Civil y el Código
de Comercio.

Ante la terminación anticipada del Contrato por cualquier causa, el
Desarrollador cesará en su preparación y el Cliente en el uso y
comercialización de los módulos o partes del Software que hayan
recibido, debiendo cumplirse, en todo caso, lo dispuesto en la
Estipulación "Obligación de secreto y confidencialidad" sobre la
obligación de confidencialidad", las "Obligaciones de no competencia" y
las restantes obligaciones aplicables.

VIGESIMOSEGUNDA. INCUMPLIMIENTO DEL CONTRATO

El incumplimiento por cualquiera de las Partes de las obligaciones
recogidas en el presente Contrato facultará a la otra Parte para, o bien
exigir su cumplimiento más el correspondiente pago de intereses
derivados del retraso en el cumplimiento, o bien resolver el Contrato en
el caso de que no se rectifique o subsane el incumplimiento por parte de
la Parte incumplidora en el plazo de diez (10) días naturales desde la
fecha en la que se verifique el incumplimiento, con la consiguiente
indemnización de daños y perjuicios más el pago de intereses por el
retraso en el cumplimiento siguiendo lo dispuesto en el artículo 1.124
del Código Civil.

Nadie podrá eximirse del cumplimiento de las obligaciones del presente
Contrato mediante el pago de la correspondiente indemnización de daños y
perjuicios, pudiendo exigirse el cumplimiento de las obligaciones o
prestaciones debidas junto a la satisfacción de la correspondiente
indemnización.

VIGESIMOTERCERA. EXIGIBILIDAD

La falta por cualquier Parte de la exigencia del cumplimiento de
cualquiera de las obligaciones recogidas en el presente Contrato no
afectará al derecho de dicha Parte a hacer valer la misma. La renuncia
por cualquier Parte de una estipulación de este Contrato no podrá
interpretarse ni como una renuncia a denunciar cualquier incumplimiento
posterior de dicha estipulación, ni como una renuncia de la misma.

VIGESIMOCUARTA. INTEGRIDAD DEL CONTRATO

Las Partes reconocen que todos los documentos Anexos y/o adjuntados al
presente Contrato forman parte integrante del mismo a todos los efectos,
y por tanto, son totalmente vinculantes para las Partes.

VIGESIMOQUINTA. PROTECCIÓN DE DATOS

Las Partes de este Contrato conocen y se obligan a cumplir el Reglamento
(UE) 2016/679 del Parlamento Europeo y del Consejo, de 27 de abril de
2016, relativo a la protección de las personas físicas en lo que
respecta al tratamiento de datos personales y a la libre circulación de
estos datos (RGPD), así como la Ley Orgánica 3/2018, de Protección de
Datos Personales y garantía de los derechos digitales y su normativa de
desarrollo, y/o aquellas que las pudieran sustituir o actualizar en el
futuro.

De esta forma, las Partes son conscientes de que mediante la firma de
este Contrato consienten que sus datos personales recogidos en el
presente Contrato, así como aquellos que se pudiesen recoger en el
futuro para poder dar cumplimiento o una correcta ejecución de este
mismo, podrían ser incorporados por la otra Parte a su propio fichero
automatizado o no de recogida de datos con el fin de ejecutar
correctamente la relación contractual y, eventualmente, para una gestión
administrativa y/o comercial.

En todo caso, las Partes se comprometen a que estos datos personales no
serán comunicados en ningún caso a terceros, aunque, si se diese el caso
de que fuera a realizarse algún tipo de comunicación de datos
personales, se comprometen siempre y de forma previa, a solicitar el
consentimiento expreso, informado, e inequívoco de la Parte que es
titular de dichos datos de carácter personal, indicando la finalidad
concreta para la que se realizará la comunicación de los datos.

De esta Estipulación no resulta ninguna limitación o restricción para
las Partes en cuanto al ejercicio de los derechos de acceso,
rectificación, supresión, limitación del tratamiento, portabilidad u
oposición con los que pudieran contar.

VIGESIMOCTAVA. LEY APLICABLE Y JURISDICCIÓN COMPETENTE

El Contrato se regirá e interpretará conforme a la legislación española
y, en particular, al Real Decreto Legislativo 1/1996, de 12 de abril,
por el que se aprueba el texto refundido de la Ley de Propiedad
Intelectual.

Las Partes se someten para la resolución de cualesquiera disputas o
reclamaciones derivadas de la interpretación o ejecución del Contrato,
incluyendo todas aquellas obligaciones no contractuales derivadas o
relativas al Contrato, a la jurisdicción de los Juzgados y Tribunales
competentes conforme a derecho.

EN VIRTUD DE LO CUAL, las Partes reconocen haber leído en su totalidad
el Contrato, manifiestan comprenderlo, y aceptan obligarse por sus
términos y condiciones, constituyendo el completo y el total acuerdo de
las Partes. Y, en prueba de conformidad, las Partes firman el presente
Contrato en todas sus hojas, y en tantas copias originales como Partes
participen en el Contrato, constituyendo todas esas copias un único
acuerdo, en el lugar y fechas indicados en el encabezamiento.
\end{document}
