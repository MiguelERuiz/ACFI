\documentclass{article}
\usepackage[utf8]{inputenc}

%%%%%%%%%%%%%%%%%%%%%%%%%%%%%%%%%%%%%%%%%%%%%%%%%%%%%%%%%%%%%%%%%%%%%%%
% Definicion de paquetes
\usepackage[T1]{fontenc}
\usepackage[spanish]{babel}
\usepackage{xargs}                      % Use more than one optional parameter in a new commands
\usepackage[pdftex,dvipsnames]{xcolor}  % Coloured text etc.

%%%%%%%%%%%%%%%%%%%%%%%%%%%%%%%%%%%%%%%%%%%%%%%%%%%%%%%%%%%%%%%%%%%%%%%
% Definición de comandos
\usepackage[colorinlistoftodos,prependcaption,textsize=tiny]{todonotes}
\newcommandx{\unsure}[2][1=]{\todo[linecolor=red,backgroundcolor=red!25,bordercolor=red,#1]{#2}}
\newcommandx{\change}[2][1=]{\todo[linecolor=blue,backgroundcolor=blue!25,bordercolor=blue,#1]{#2}}
\newcommandx{\info}[2][1=]{\todo[linecolor=OliveGreen,backgroundcolor=OliveGreen!25,bordercolor=OliveGreen,#1]{#2}}
\newcommandx{\improvement}[2][1=]{\todo[linecolor=Plum,backgroundcolor=Plum!25,bordercolor=Plum,#1]{#2}}
\newcommandx{\thiswillnotshow}[2][1=]{\todo[disable,#1]{#2}}

%%%%%%%%%%%%%%%%%%%%%%%%%%%%%%%%%%%%%%%%%%%%%%%%%%%%%%%%%%%%%%%%%%%%%%%
% Definición de portada
\title{DescripcionTrabajo}
\author{amaliare }
\date{October 2021}

%%%%%%%%%%%%%%%%%%%%%%%%%%%%%%%%%%%%%%%%%%%%%%%%%%%%%%%%%%%%%%%%%%%%%%%
%% Empieza el documento
\begin{document}
\improvement{MR: añadir portada, cambiar titulos y autores, poner fecha}
\section*{Descripcion Trabajo ACFI}

En primer lugar, todos los integrantes del grupo discutimos qué aplicación
queríamos y qué funcionalidades iba a tener. Una vez decididos todos los
requisitos de la aplicación, Amalia y Sara actuaron como cliente y,
Sergio, Miguel y José Luis como desarrolladores.

En su rol de cliente, Amalia y Sara les explicaron a los desarrolladores del
otro equipo la idea que habíamos tenido y sin entrar en detalles de
la implementación que se requería. Al mismo tiempo Sergio, Miguel y José Luis
en su rol de desarrollador, escucharon y entendieron lo que nuestro cliente
querían como producto.

Posteriormente, elaboramos un plan para repartirnos las tareas basándonos en las
piezas que más trabajo requerían y nos dividimos en dos grupos en dos fases: en
la primera fase, se realizó el Diagrama de Gantt por parte de José Luis, Sara
y Amalia, mientras que Sergio y Miguel elaboraron la matriz de riegos.
Tras acabar dicha fase, Sergio, Amalia y Sara calcularon el presupuesto, al
mismo tiempo que José Luis y Miguel revisaron que en  el borrador del contrato
se especificaban todas las especificaciones que el cliente nos había demandado.

Por otro lado, recibimos el contrato del equipo que se encargaba de realizarlo.
Tras haber sido leído por todos los miembros del equipo, informamos al equipo de
los puntos que estábamos en desacuerdo o que creíamos que había que matizar.
Cuando se lo comunicamos, nos explicaron ciertos apartados y otros fueron
corregidos para reflejar lo que queríamos.

\improvement{MR: Esta parte no la termino de ver.}

Finalmente, ya con todo el material, se terminó de maquetar el contrato y fue enviado.
Cuando recibimos la respuesta del cliente hicimos las últimas correcciones del
proyecto, que finalmente fue firmado por el cliente.

\end{document}
