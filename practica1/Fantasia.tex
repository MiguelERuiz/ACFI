\documentclass{report}
\usepackage[left=2cm,top=2.5cm,right=2cm,bottom=2.5cm]{geometry}


\usepackage[T1]{fontenc}
\usepackage[utf8]{inputenc}
\usepackage[spanish]{babel}
\usepackage{enumitem} % Listas enumeradas alfabéticamente
\usepackage{eurosym}
\usepackage[colorinlistoftodos,prependcaption,textsize=tiny]{todonotes}



\begin{document}
    \title{Aplicación fantasía}
	\author{
		Sergio García Sánchez
		\and
		Sara Juberías Campos
		\and
		José Luis Mela Navarro
		\and
		Amalia Regueira Fernández
		\and
		Miguel Emilio Ruiz Nieto
	}

	\maketitle
	\section*{Aplicación Fantasía}
	\begin{itemize}
		\item Aplicación web para retransmitir conciertos y vender entradas (físicas y online)
		\item Aplicación para la tele y para el móvil
		\item Clientes:
		\begin{itemize}
			\item Asistente conciertos
			\item Usuarios que ofrecen cosas (promotor de conciertos y vendedor de merchandising)
		\end{itemize}
		\item Plataforma de pago
		\item Plataforma de registro de usuarios
		\item Plataforma e-mail
		\item Buscador de conciertos (ubicación exacta, género, precio, artista)
		\item Un chat público para cada retransmisión
		\item Chat entre usuarios (privado)
		\item Elección de cámara en la transmisión
		\item Publicidad
		\item Subtítulos en muchos idiomas con inteligencia artificial
		\item Verde
		\item Rápida
		\item Guardar las grabaciones del concierto para que esten disponibles posteriormente previo pago
		\item Que se parezca a twitch (pero no mucho)
		\item Que funcione bien
		\item Que sea bonita
		\item Que tenga calidad
		\item Pagar con bitcoins, paypal, con tarjeta, con klarna
		\item Que no se caiga (\{textbf{y mucho menos en medio de un concierto})
		\item Retransmisiones simultáneas
	\end{itemize}
\end{document}