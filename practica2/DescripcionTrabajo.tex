\documentclass[a4paper,12pt]{report}

%%%%%%%%%%%%%%%%%%%%%%%%%%%%%%%%%%%%%%%%%%%%%%%%%%%%%%%%%%%%%%%%%%%%%%%
% Definicion de paquetes
\usepackage[T1]{fontenc}
\usepackage[utf8]{inputenc}
\usepackage[spanish]{babel}
\usepackage{xargs}                      % Use more than one optional parameter in a new commands
\usepackage[pdftex,dvipsnames]{xcolor}  % Coloured text etc.

%%%%%%%%%%%%%%%%%%%%%%%%%%%%%%%%%%%%%%%%%%%%%%%%%%%%%%%%%%%%%%%%%%%%%%%
% Definición de comandos
\usepackage[colorinlistoftodos,prependcaption,textsize=tiny]{todonotes}
\newcommandx{\unsure}[2][1=]{\todo[linecolor=red,backgroundcolor=red!25,bordercolor=red,#1]{#2}}
\newcommandx{\change}[2][1=]{\todo[linecolor=blue,backgroundcolor=blue!25,bordercolor=blue,#1]{#2}}
\newcommandx{\info}[2][1=]{\todo[linecolor=OliveGreen,backgroundcolor=OliveGreen!25,bordercolor=OliveGreen,#1]{#2}}
\newcommandx{\improvement}[2][1=]{\todo[linecolor=Plum,backgroundcolor=Plum!25,bordercolor=Plum,#1]{#2}}
\newcommandx{\thiswillnotshow}[2][1=]{\todo[disable,#1]{#2}}

%%%%%%%%%%%%%%%%%%%%%%%%%%%%%%%%%%%%%%%%%%%%%%%%%%%%%%%%%%%%%%%%%%%%%%%
%% Empieza el documento
\begin{document}

\title{Desarrollo del Trabajo}

\author{
		Sergio García Sánchez
		\and
		Sara Juberías Campos
		\and
		José Luis Mela Navarro
		\and
		Amalia Regueira Fernández
		\and
		Miguel Emilio Ruiz Nieto
	}

\maketitle
\section*{Desarrollo del trabajo}

En primer lugar, los miembros de grupo comenzamos a leer la documentación aportada para empezar a
comprender la naturaleza y el alcance del propio proyecto. Esta tarea la hicimos todos a excepción
de Sergio, que dedicó su esfuerzo en intentar ejecutar la aplicación sin éxito. Dado que no queríamos
invertir demasiado tiempo en ello, optamos por dejar de lado el arranque de la aplicación y dividirnos el
trabajo. Dicha división fue en base a la tabla de la sección Incumplimientos y en dos equipos:
por un lado, Amalia, Sergio y Sara; y por otro, José Luis y Miguel.

Cuando comenzamos a redactar el informe de auditoría, Sara junto con Amalia empezaron a estructurar
el documento de entrega y completaron las secciones de Destinatario, Entidad Auditada, Comparabilidad
y Salvedades.

A partir de ese momento, fuimos repartiéndonos los distintos objetivos de la tabla por equipos,
y en función de que un equipo acabara el punto que le correspondía, se abordaba el siguiente.
En primera instancia, el primer equipo se asignó el primer punto y el segundo equipo, el segundo.
Mientras que el segundo equipo le llevó más tiempo que al segundo completar los apartados del Plan de desarrollo, el primer equipo fue mucho más rápido y desarrolló más objetivos. Dicho lo cual, el primer
equipo desarrolló los objetivos 1, 3, 5, 6 y 7, mientras que el segundo equipo solo desarrolló los puntos
2 y 4.

Una vez acabado el trabajo relacionado con justificar los incumplimientos, discutimos acerca de cuáles
de estos iban a ser más relevantes de cara a desarrollar el Énfasis. El resto del documento fue abordado
por todos los miembros del equipo.

\end{document}
